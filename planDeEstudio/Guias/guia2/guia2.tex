\documentclass{article}

%### Hacer uso de símbolos extra ###
\usepackage{latexsym,amsmath,amssymb,amsfonts}

%### Cambio de la fuente del documento###
\usepackage{mathpazo} %palatino

%### Incluir Graphicos ###
\usepackage{graphicx}

%### Comandos especiales para TABLAS ###
\usepackage{multirow, bigstrut}

%### Hiper referencias y ocultando el link (hidelinks) ###
\usepackage[hidelinks]{hyperref}

%##########################################
%Formato archivo.tex, entradas de teclado
%solo dejar uno de los dos archivos activados
\usepackage[utf8]{inputenc}
%\usepackage[latin1]{inputenc}
%###########################################
\usepackage[T1]{fontenc}

%Título
\title{Práctica 2\\Interruptor Eléctrico}

%Author
\author{Cubides Castro, Johnny German \\
        Pulido Gómez, Carolina Rosa\\Murcia Narváez, Julián Andrés}
%### Agregar fecha manualmente ###
%\date{mmm, dia, año}

\begin{document}
\maketitle
\section{¿ Qué es el transistor ?}

\section{Modos de operación} 
\subsection{Como interruptor}
\begin{itemize}
\item Entradas, salidas, fuentes, conectores, drivers.
\item Explicar lógica booleana uno y cero.
\item ¿ De donde viene el uno?, ¿ qué significa?.
\item ¿ Qué es una señal?.
\item ¿ Cuál es la señal de control?, ¿ Cuál es la señal de potencia?. 
\item ¿ Qué características tienen las señales?.
\end{itemize}
\subsection{Como variador de intensidad lumínica}
\begin{itemize}
\item Identificar entradas, salidas, conectores, drivers, fuentes.
\item ¿ Qué es una ganancia ?.
\end{itemize}
\subsection{El motor y el cambio de giro}
\begin{itemize}
\item ¿ A qué se debe el sentido de la corriente?.
\end{itemize}
\subsection{Como variador de velocidad}
\begin{itemize}
\item ¿ A qué se debe el cambio de la velocidad del motor?.
\end{itemize}







%\include{dir/arch}
\end{document}
