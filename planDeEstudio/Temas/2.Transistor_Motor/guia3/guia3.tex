\documentclass{article}

%### Hacer uso de símbolos extra ###
\usepackage{latexsym,amsmath,amssymb,amsfonts}

%### Cambio de la fuente del documento###
\usepackage{mathpazo} %palatino

%### Incluir Graphicos ###
\usepackage{graphicx}

%### Comandos especiales para TABLAS ###
\usepackage{multirow, bigstrut}

%### Hiper referencias y ocultando el link (hidelinks) ###
\usepackage[hidelinks]{hyperref}

%##########################################
%Formato archivo.tex, entradas de teclado
%solo dejar uno de los dos archivos activados
\usepackage[utf8]{inputenc}
%\usepackage[latin1]{inputenc}
%###########################################
\usepackage[T1]{fontenc}

%Título
\title{Práctica 3\\Circuitos y transistores}

%Author
\author{Cubides Castro, Johnny German \\
        Pulido Gómez, Carolina Rosa\\Murcia Narváez, Julián Andrés}
%### Agregar fecha manualmente ###
%\date{mmm, dia, año}

\begin{document}
\maketitle
\section{Manejo de la Protoboard y sentido de la corriente (1 HORA) }
\begin{itemize}
\item[1] Lineas Horizontales y verticales(Horizontales-Fuentes, Verticales-componentes).
\item[2] ¿ Cómo no dañar los componentes?.
\item Terminales (Transistor y potenciometro).
\item Resistecia + led (¿ Por qué ?).
\item ¿ Qué es un corto?(Seguridad).
\item Errores comunes.
\item[3]¿ tiene sentido la corriente?.
\item[4]¿ Cómo lo vemos? (EN LED).
\item[5]¿ Identificar la polaridad de los componentes?(resistencia no tiene polaridad).
\item[6]¿ El motor Tiene polaridad?(sentido de la corriente)
\end{itemize}

\section{Transistor como amplificador de intensidad y potenciometro como variador de Intensidad (1 HORA)}
\begin{itemize}
\item[1] Magnitud de corriente (flechas).
\item[2] Escala con análogos----> Potenciometro(llave),Transistor(Multiplicador).
\item[3] Identificar lo que hace el driver (circuito led-potenciometro-transistor).
. 
\end{itemize}






%\include{dir/arch}
\end{document}
