\documentclass{article}

\usepackage[papersize={215mm,279mm},tmargin=10mm, 
bmargin=10mm,lmargin=10mm, rmargin=10mm]{geometry}

%### Hacer uso de símbolos extra ###
\usepackage{latexsym,amsmath,amssymb,amsfonts}

%### Cambio de la fuente del documento###
\usepackage{mathpazo} %palatino

%### Incluir Graphicos ###
\usepackage{graphicx}

%### Comandos especiales para TABLAS ###
\usepackage{multirow, bigstrut}

%### Hiper referencias y ocultando el link (hidelinks) ###
\usepackage{hyperref}

%##########################################
%Formato archivo.tex, entradas de teclado
%solo dejar uno de los dos archivos activados
\usepackage[utf8]{inputenc}
%\usepackage[latin1]{inputenc}
%###########################################
\usepackage[T1]{fontenc}

%##########################################
% Separar palarbras correctamente.
\hyphenation{co-rres-pon-da}

%Título
\title{Guia\\Subtítulo}

%Author
\author{Grupo Pingüino Tux}
%### Agregar fecha manualmente ###
%\date{mmm, dia, año}

\begin{document}
\maketitle
%\include{dir/arch}

\section{Objetivo}
\subsection{General}
Orientar al estudiante para que tenga la capacidad de abstraer del mundo
real algún evento o cosa y manifestar su comportamiento en un modelo que 
corresponda a la lógica booleana.

\subsection{Específicos}

\begin{itemize}

\item Ilustrar las funciones lógicas (booleanas) con sus respectivas
		tablas de verdad.

\item Usar un software que permita la simulación del comportamiento de
		las funciones lógicas.

\item Abstraer de un circuito eléctrico la lógica digital
para luego ilustrarla en compuertas lógicas booleanas. 

\end{itemize}

\section{Procedimiento}

\subsection{Compuertas lógicas}

Se debe mostrar cada una de las compuertas lógicas con sus respectivas
tablas de verdad, iniciando con las que manejan una sola entrada, es
decir, BUFFER (seguidor), NOT (negador), para luego continuar con la
las compuertas de dos entradas como lo son: OR (sumador), AND 
(multplicador), XOR, además, mostrar la combinación de las las compuertas 
de dos o más entradas con la compuerta negadora que genera los demás casos,
como son: NOR, NAND, XNOR.

\subsection{Logisim}

\href{http://www.cburch.com/logisim/index_es.html}{Logisim}\footnote{De la
		misma página oficial se puede obtener un tutorial de inicio. Visitar
		el siguiente \href{http://www.cburch.com/logisim/docs/2.1.0-es/guide/tutorial/index.html}{link}} permimte 
realizar simulaciones de sistemas secuenciales como 
combinacionales. Pueden simulase entradas  como teclado y 
mandos. Se debe tener instalado la máquina virtual de java del S.O. en uso
para poder correr el archivo .jar.

Se debe de reconocer las compuertas lógicas a través de su símbolos
esquemáticos. Mostrar las entradas y salidas posibles en el software para
su rápida asociación con los ejemplos electrónicos (mundo real).

\subsection{Abstracción de circuitos}

\end{document}
\grid
\grid
\grid
