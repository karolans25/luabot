\section{Blockly LuaBot}

\subsection{Insertar Categoría en la caja de herramientas} 

Para insertar una nueva categoría en la caja de herramientas de blockly 
basta con agregar las siguientes líneas en el fichero \textit{index.html}

\begin{minted}[bgcolor=white]{XML}
<category name=``nueva_categoria''>
	... Contenido ...
</category>
\end{minted}

Donde \textbf{nueva\_categoria} se sustituye por la categoría que se está
creando.

Ejemplo: en el \textit{index.html} se agregará una nueva categoría llamada
Pines

\begin{minted}[bgcolor=white]{HTML}
<!DOCTYPE html>
<html>
<head>
... Contenido ...
</head>
<body>
... Contenido ...
<xml id=``toolbox'' style=``display: none''>
... Contenido ...
<sep></sep>
<!-- A continuacion la nueva categoria -->
<category name=``Pines''>
	... Contenido ...
</category>
</xml>
</body>
</html>
\end{minted}


\subsection{Insertar Herramienta en Categoría}

Para insertar una herramienta basta con agregar la siguiente línea:

\begin{minted}[bgcolor=white]{XML}
<block type=``nueva_herramienta''></block>
\end{minted}

Donde \textbf{nueva\_herramienta} se sustituye por el nombre de la 
herramienta creada con \textit{blocklyFactory}

Ejemplo: En el \textit{index.html} se agregará la herramienta gpio\_mode 
en la categoría Pines.

\begin{minted}[bgcolor=white]{HTML}
<!DOCTYPE html>
<html>
<head>
... Contenido ...
</head>
<body>
... Contenido ...
<xml id=``toolbox'' style=``display: none''>
... Contenido ...
<sep></sep>
<category name=``Pines''>
<!-- A continuacion la nueva herramienta -->
	<block type=``gpio_mode''></block>
</category>
</xml>
</body>
</html>
\end{minted}

\subsection{Crear Nueva Herramienta con BlocklyFactory}

\textit{BlocklyFactory} Permite crear nuevos bloques (las herramientas) 
de una manera gráfica.

Como se puede ver en la imagen \ref{fig:blocklyfactory}

\onecolumn

%Imagen para IEEE
\begin{figure}[hptp]
    \centering
    \includegraphics[scale=0.4]{imag/blocklyfactory.png}
    \caption{BlocklyFactory}
    \label{fig:blocklyfactory}
\end{figure}
\smallskip

\twocolumn
